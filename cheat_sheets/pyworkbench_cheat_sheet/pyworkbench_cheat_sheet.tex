\documentclass[9pt,landscape]{article}
\usepackage{{./../static/style}}
\pdfinfo{
  /Title (PyWorkbench cheat sheet)
  /Creator (TeX)
  /Producer (pdfTeX 1.40.0)
  /Author (Ansys)
  /Subject (PyWorkbench)
  /Keywords (PyAnsys, PyWorkbench, Workbench, cheat sheet, template)}

\begin{document}
\raggedright
\footnotesize

% Add the title of cheat sheet here
% ----------------------------------------
\begin{center}
     \Huge{\textbf{PyWorkbench cheat sheet}} \\
\end{center}

\begin{center}
  \small{\textbf{Version: 0.5 (stable) - API is subject to change}} \\
\end{center}

\AddToShipoutPicture*
  {\put(670,577.5){\includegraphics[height = 1.2cm]{ansys.png}}}
\AddToShipoutPictureBG*{\includegraphics[width=\paperwidth]{bground.png}}
\vspace{-0.15cm}
\noindent\makebox[\linewidth]{\rule{\paperwidth}{2pt}}

\begin{multicols}{3}
\setlength{\premulticols}{1pt}
\setlength{\postmulticols}{1pt}
\setlength{\multicolsep}{1pt}
\setlength{\columnsep}{2pt}

% session starts here. 
% First colomn
% --------------------------------------------------------------------------------
\vfill
\section{\includegraphics[height=\fontcharht\font`\S]{slash.png} # Connect PyWorkbench to Ansys Workbench from Python}

% \vspace{5mm} %5mm vertical space
% {\color{orange} \rule{\linewidth}{0.1mm}}

\section{\includegraphics[height=\fontcharht\font`\S]{slash.png} Local connection to a Workbench Server}
Execute the steps below to link PyWorkbench with a local Ansys Workbench session:

\begin{itemize}
\item Initiate Ansys Workbench  
\item Input StartServer() in the Workbench Command Window 
\item Utilize the given port number to link PyWorkbench with the server
\end{itemize}    

\pythoncode{scripts/generated_scripts/pyworkbench_script_0.py}

\section{\includegraphics[height=\fontcharht\font`\S]{slash.png} Remote connection to a Workbench Server}
Execute the steps below to link PyWorkbench with a local Ansys Workbench session:
\pythoncode{scripts/generated_scripts/pyworkbench_script_1.py}

\section{\includegraphics[height=\fontcharht\font`\S]{slash.png} Launch Workbench Server and Start a Client}
You can initiate a Workbench server and activate a client through a Python script on the client-side.

This script initiates a server on a local Windows-based system:
\pythoncode{scripts/generated_scripts/pyworkbench_script_2.py}

This script initiates a server on a remote Windows device using appropriate user authentication:
\pythoncode{scripts/generated_scripts/pyworkbench_script_3.py}

% row 2 col 1
\section{\includegraphics[height=\fontcharht\font`\S]{slash.png} Execute scripts on the Workbench server}

These methods can be utilized to execute IronPython-based Workbench scripts containing commands or queries, with the help of PyWorkbench:

\begin{itemize}
\item \texttt{run_script_string()}: Executes a script that is included within the provided string
\item \texttt{run_script_file()}: Executes a script file in the client's working directory
\end{itemize}    

Using \texttt{run_script_string()} method:
\pythoncode{scripts/generated_scripts/pyworkbench_script_4.py}

Using \texttt{run_script_file()} method:
\pythoncode{scripts/generated_scripts/pyworkbench_script_5.py}

You have the ability to assign any output that needs to be returned from these methods to the global variable \texttt{wb_script_result} in the script, as a JSON string.
Workbench script returns all message summaries from the Workbench session:
\pythoncode{scripts/generated_scripts/pyworkbench_script_6.py}

Below script while execution will display \texttt{info}, \texttt{warning}, and \texttt{error} levels in the logger.
\pythoncode{scripts/generated_scripts/pyworkbench_script_7.py}

% Second column
% --------------------------------------------------------------------------------
% row 1 col 2

\vfill

\section{\includegraphics[height=\fontcharht\font`\S]{slash.png} Upload and download files}

Using the \texttt{upload_file()} and \texttt{download_file()} methods, you have the ability to transfer data files to the server (upload) and retrieve them from the server (download).
You can use the GetServerWorkingDirectory() query in the scripts that execute on the server to retrieve the server's operating directory.

This script uploads all *.prt files and *.agdb files in the working directory, and another file from a different directory, from the client to the server:
\pythoncode{scripts/generated_scripts/pyworkbench_script_8.py}

\vspace{2mm} %5mm vertical space

This server-side script loads a geometry file into a new Workbench system from the server's directory:

\pythoncode{scripts/generated_scripts/pyworkbench_script_9.py}

This server-side script transfers a Mechanical solver output file to the server's directory from Workbench:

\pythoncode{scripts/generated_scripts/pyworkbench_script_10.py}

This client script retrieves all .out files from the server's working directory:

\pythoncode{scripts/generated_scripts/pyworkbench_script_11.py}

Using the \texttt{download_project_archive()} function you can save your current Workbench project on the server, archive it, and then download the archived project to the client:

\pythoncode{scripts/generated_scripts/pyworkbench_script_12.py}


% Third column
% --------------------------------------------------------------------------------
% \vfill
% row 1 col 3

\section{\includegraphics[height=\fontcharht\font`\S]{slash.png} Initiate additional PyAnsys services for systems within a Workbench project}
% \vspace{5mm} %5mm vertical space
% {\color{orange} \rule{\linewidth}{0.1mm}}

\section{\includegraphics[height=\fontcharht\font`\S]{slash.png} PyMechanical}

In a Workbench project, you can operate and link the PyMechanical service from the same client machine. 
This script creates a mechanical system server-side, then starts the PyMechanical service and client:
\pythoncode{scripts/generated_scripts/pyworkbench_script_13.py}

\section{\includegraphics[height=\fontcharht\font`\S]{slash.png} PyFluent}

This script initiates the PyFluent service along with the client for a Fluent system that was developed in Workbench:
\pythoncode{scripts/generated_scripts/pyworkbench_script_14.py}

\section{\includegraphics[height=\fontcharht\font`\S]{slash.png} PySherlock}

This script initiates the PySherlock service and its client for a Sherlock system set up in Workbench:
\pythoncode{scripts/generated_scripts/pyworkbench_script_15.py}

% Add subsection
% This section includes useful links to the documentation.
% Examples: installation, API reference, commands, examples.
% Replace 'name of link' with appropriate display text.

\subsection{References from PyWorkbench documentation}
\begin{multicols}{2}
\begin{itemize}
    \item \href{https://workbench.docs.pyansys.com/version/stable/getting-started.html}{\color{blue}{Getting started}}
    \item \href{https://workbench.docs.pyansys.com/version/stable/user-guide.html}{\color{blue}{User guide}}
    \item \href{https://workbench.docs.pyansys.com/version/stable/examples.html}{\color{blue}{Examples}}
    \item \href{https://workbench.docs.pyansys.com/version/stable/api/index.html}{\color{blue}{API reference}}
\end{itemize}
\end{multicols}
\end{multicols}

% Footer session of the latex with link to documentation and GitHub page
\vspace{-0.15cm}
\noindent\makebox[\linewidth]{\rule{\paperwidth}{4pt}}
\begin{center}
Getting started with PyWorkbench \includegraphics[height=\fontcharht\font`\S]{slash.png} \href{https://github.com/ansys/pyworkbench}{\color{blue}{pyworkbench on GitHub}}}
Visit \includegraphics[height=\fontcharht\font`\S]{slash.png} \href{https://workbench.docs.pyansys.com/version/stable/}{\color{blue}{workbench.docs.pyansys.com}}} 
\end{center}
\end{document}
